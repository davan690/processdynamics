\documentclass[]{book}
\usepackage{lmodern}
\usepackage{amssymb,amsmath}
\usepackage{ifxetex,ifluatex}
\usepackage{fixltx2e} % provides \textsubscript
\ifnum 0\ifxetex 1\fi\ifluatex 1\fi=0 % if pdftex
  \usepackage[T1]{fontenc}
  \usepackage[utf8]{inputenc}
\else % if luatex or xelatex
  \ifxetex
    \usepackage{mathspec}
  \else
    \usepackage{fontspec}
  \fi
  \defaultfontfeatures{Ligatures=TeX,Scale=MatchLowercase}
\fi
% use upquote if available, for straight quotes in verbatim environments
\IfFileExists{upquote.sty}{\usepackage{upquote}}{}
% use microtype if available
\IfFileExists{microtype.sty}{%
\usepackage{microtype}
\UseMicrotypeSet[protrusion]{basicmath} % disable protrusion for tt fonts
}{}
\usepackage{hyperref}
\hypersetup{unicode=true,
            pdftitle={Diagrams},
            pdfauthor={Anthony Davidson},
            pdfborder={0 0 0},
            breaklinks=true}
\urlstyle{same}  % don't use monospace font for urls
\usepackage{natbib}
\bibliographystyle{apalike}
\usepackage{color}
\usepackage{fancyvrb}
\newcommand{\VerbBar}{|}
\newcommand{\VERB}{\Verb[commandchars=\\\{\}]}
\DefineVerbatimEnvironment{Highlighting}{Verbatim}{commandchars=\\\{\}}
% Add ',fontsize=\small' for more characters per line
\usepackage{framed}
\definecolor{shadecolor}{RGB}{248,248,248}
\newenvironment{Shaded}{\begin{snugshade}}{\end{snugshade}}
\newcommand{\AlertTok}[1]{\textcolor[rgb]{0.94,0.16,0.16}{#1}}
\newcommand{\AnnotationTok}[1]{\textcolor[rgb]{0.56,0.35,0.01}{\textbf{\textit{#1}}}}
\newcommand{\AttributeTok}[1]{\textcolor[rgb]{0.77,0.63,0.00}{#1}}
\newcommand{\BaseNTok}[1]{\textcolor[rgb]{0.00,0.00,0.81}{#1}}
\newcommand{\BuiltInTok}[1]{#1}
\newcommand{\CharTok}[1]{\textcolor[rgb]{0.31,0.60,0.02}{#1}}
\newcommand{\CommentTok}[1]{\textcolor[rgb]{0.56,0.35,0.01}{\textit{#1}}}
\newcommand{\CommentVarTok}[1]{\textcolor[rgb]{0.56,0.35,0.01}{\textbf{\textit{#1}}}}
\newcommand{\ConstantTok}[1]{\textcolor[rgb]{0.00,0.00,0.00}{#1}}
\newcommand{\ControlFlowTok}[1]{\textcolor[rgb]{0.13,0.29,0.53}{\textbf{#1}}}
\newcommand{\DataTypeTok}[1]{\textcolor[rgb]{0.13,0.29,0.53}{#1}}
\newcommand{\DecValTok}[1]{\textcolor[rgb]{0.00,0.00,0.81}{#1}}
\newcommand{\DocumentationTok}[1]{\textcolor[rgb]{0.56,0.35,0.01}{\textbf{\textit{#1}}}}
\newcommand{\ErrorTok}[1]{\textcolor[rgb]{0.64,0.00,0.00}{\textbf{#1}}}
\newcommand{\ExtensionTok}[1]{#1}
\newcommand{\FloatTok}[1]{\textcolor[rgb]{0.00,0.00,0.81}{#1}}
\newcommand{\FunctionTok}[1]{\textcolor[rgb]{0.00,0.00,0.00}{#1}}
\newcommand{\ImportTok}[1]{#1}
\newcommand{\InformationTok}[1]{\textcolor[rgb]{0.56,0.35,0.01}{\textbf{\textit{#1}}}}
\newcommand{\KeywordTok}[1]{\textcolor[rgb]{0.13,0.29,0.53}{\textbf{#1}}}
\newcommand{\NormalTok}[1]{#1}
\newcommand{\OperatorTok}[1]{\textcolor[rgb]{0.81,0.36,0.00}{\textbf{#1}}}
\newcommand{\OtherTok}[1]{\textcolor[rgb]{0.56,0.35,0.01}{#1}}
\newcommand{\PreprocessorTok}[1]{\textcolor[rgb]{0.56,0.35,0.01}{\textit{#1}}}
\newcommand{\RegionMarkerTok}[1]{#1}
\newcommand{\SpecialCharTok}[1]{\textcolor[rgb]{0.00,0.00,0.00}{#1}}
\newcommand{\SpecialStringTok}[1]{\textcolor[rgb]{0.31,0.60,0.02}{#1}}
\newcommand{\StringTok}[1]{\textcolor[rgb]{0.31,0.60,0.02}{#1}}
\newcommand{\VariableTok}[1]{\textcolor[rgb]{0.00,0.00,0.00}{#1}}
\newcommand{\VerbatimStringTok}[1]{\textcolor[rgb]{0.31,0.60,0.02}{#1}}
\newcommand{\WarningTok}[1]{\textcolor[rgb]{0.56,0.35,0.01}{\textbf{\textit{#1}}}}
\usepackage{longtable,booktabs}
\usepackage{graphicx,grffile}
\makeatletter
\def\maxwidth{\ifdim\Gin@nat@width>\linewidth\linewidth\else\Gin@nat@width\fi}
\def\maxheight{\ifdim\Gin@nat@height>\textheight\textheight\else\Gin@nat@height\fi}
\makeatother
% Scale images if necessary, so that they will not overflow the page
% margins by default, and it is still possible to overwrite the defaults
% using explicit options in \includegraphics[width, height, ...]{}
\setkeys{Gin}{width=\maxwidth,height=\maxheight,keepaspectratio}
\IfFileExists{parskip.sty}{%
\usepackage{parskip}
}{% else
\setlength{\parindent}{0pt}
\setlength{\parskip}{6pt plus 2pt minus 1pt}
}
\setlength{\emergencystretch}{3em}  % prevent overfull lines
\providecommand{\tightlist}{%
  \setlength{\itemsep}{0pt}\setlength{\parskip}{0pt}}
\setcounter{secnumdepth}{5}
% Redefines (sub)paragraphs to behave more like sections
\ifx\paragraph\undefined\else
\let\oldparagraph\paragraph
\renewcommand{\paragraph}[1]{\oldparagraph{#1}\mbox{}}
\fi
\ifx\subparagraph\undefined\else
\let\oldsubparagraph\subparagraph
\renewcommand{\subparagraph}[1]{\oldsubparagraph{#1}\mbox{}}
\fi

%%% Use protect on footnotes to avoid problems with footnotes in titles
\let\rmarkdownfootnote\footnote%
\def\footnote{\protect\rmarkdownfootnote}

%%% Change title format to be more compact
\usepackage{titling}

% Create subtitle command for use in maketitle
\providecommand{\subtitle}[1]{
  \posttitle{
    \begin{center}\large#1\end{center}
    }
}

\setlength{\droptitle}{-2em}

  \title{Diagrams}
    \pretitle{\vspace{\droptitle}\centering\huge}
  \posttitle{\par}
    \author{Anthony Davidson}
    \preauthor{\centering\large\emph}
  \postauthor{\par}
      \predate{\centering\large\emph}
  \postdate{\par}
    \date{2019-10-14}

\usepackage{booktabs}

\begin{document}
\maketitle

{
\setcounter{tocdepth}{1}
\tableofcontents
}
\hypertarget{exams}{%
\chapter{Examples}\label{exams}}

Here are a collection of diagrams that have extended explainations (in time) within the pages.

\hypertarget{intro}{%
\chapter{Introduction}\label{intro}}

Conceptial diagrams are vital for understanding processes and developing incremetal science, however, there seems to be very few packages or other software that aligns with the literature in how and when these diagrams should be developed.

This book uses \texttt{R} and the core packages \href{https://cran.r-project.org/web/packages/DiagrammeR/index.html}{\texttt{diagrammeR}}.

There is a great resource written by the author \href{https://rich-iannone.github.io/DiagrammeR/index.html}{here} but I am still working out overall how the package in R works.
- A good explaination of the features \href{https://rich-iannone.github.io/DiagrammeR/index.html\#features}{here}

\hypertarget{resources}{%
\section{Resources}\label{resources}}

The general concept of the package I am using is very simple but the code looks intemidating:

\begin{Shaded}
\begin{Highlighting}[]
\KeywordTok{grViz}\NormalTok{(}\StringTok{"}
\StringTok{digraph boxes_and_circles \{}

\StringTok{  # a 'graph' statement}
\StringTok{  graph [overlap = true, fontsize = 10]}

\StringTok{  # several 'node' statements}
\StringTok{  node [shape = box,}
\StringTok{        fontname = Helvetica]}
\StringTok{  'box'; 'shape'}

\StringTok{  node [shape = circle,}
\StringTok{        fixedsize = true,}
\StringTok{        width = 0.9] // sets as circles}
\StringTok{  'shape2'; 'circles'}

\StringTok{  node [shape = circle,}
\StringTok{        fixedsize = true,}
\StringTok{        width = 0.9] // sets as circles}
\StringTok{  'join'; 'arrow'}
\StringTok{  }
\StringTok{  # several 'edge' statements}
\StringTok{  join->arrow}
\StringTok{\}}
\StringTok{"}\NormalTok{)}
\end{Highlighting}
\end{Shaded}

\includegraphics{Simple_process_diagrams_files/figure-latex/unnamed-chunk-5-1.pdf}

\hypertarget{my-notes}{%
\section{My notes}\label{my-notes}}

For example, you might like to think of a simple statistical regression with a single \texttt{y} and \texttt{x}as:

\[ y_i = \beta_0 + \beta_1(x_i) + error_i \]
And if this was to be diagramatically represented it could be describled like so with the following descriptions:

\emph{1:} We are proposing that \(y_i\) is directly effected by \(x_i\).

\includegraphics{Simple_process_diagrams_files/figure-latex/unnamed-chunk-6-1.pdf}

\hypertarget{examples}{%
\subsection{Examples}\label{examples}}

This info I think is interesting but haven't investigated it much yet.

\hypertarget{math-insert}{%
\subsection{Math insert}\label{math-insert}}

Here is an example of inserting math equations into \texttt{diagrammeR} using mathJax??

\begin{verbatim}
## 
## > library(DiagrammeR)
## 
## > add_mathjax(grViz("\ndigraph G {\n  A [label=\"$\\\\cos (2\\\\theta) = \\\\cos^2 \\\\theta - \\\\sin^2 \\\\theta$\"];\n  A -> B;\n}\n"))
\end{verbatim}

\hypertarget{kol-but-not-working}{%
\subsection{Kol but not working}\label{kol-but-not-working}}

\hypertarget{mixed-models}{%
\subsection{Mixed models}\label{mixed-models}}

Mixed models essentially start with sample mean differences between groups and then add additional complexity in favour of poor fitting models.

You can write out models and variable relationships using statistical equations such as:

\[ outcome = \beta_0 + \beta_1(sex_i)\]
where the variables can be collected either through direct questions or transformed data from the orginal questions. What does it look like in a diagram explaining how the variables are proposed to be similar? Example below:

For example you might think that generally individuals that \ldots{}

\[ outcome = \beta_0 + \beta_1(sex_i) + \beta(age_i) + error_i \]

\hypertarget{comps}{%
\chapter{Software and Hardware}\label{comps}}

This is a great question I have been playing around with to help learn how to use the \texttt{diagrammeR} package. The question is:
\textbf{What is the difference between software and hardware?}

I thought that is was an obvious difference. Hardware is the stuff that makes the software work.

For example, \texttt{RAM} is hardware, \texttt{word} is software?

\begin{longtable}[]{@{}ll@{}}
\toprule
Software & Hardware\tabularnewline
\midrule
\endhead
R & RAM\tabularnewline
word & CPU\tabularnewline
excel & Processors\tabularnewline
\bottomrule
\end{longtable}

\hypertarget{software}{%
\section{Software}\label{software}}

But the problem is when somebody knows the interface (software that uses many types of software). Examples of this are programs like \texttt{RStudio}, \texttt{vs-code} and other such programs. This is because they are a single piece of software that uses ``heaps'' of other software.

\includegraphics{Simple_process_diagrams_files/figure-latex/unnamed-chunk-13-1.pdf}

But then each bit of software such as \texttt{GIT} and \texttt{R} can be accessed by many different types of software\ldots so is \texttt{R} not just hardware then?

\includegraphics{Simple_process_diagrams_files/figure-latex/unnamed-chunk-14-1.pdf}

And now the question becomes about the definition of what software and hardware are\ldots{}

\begin{longtable}[]{@{}ll@{}}
\toprule
Software & Hardware\tabularnewline
\midrule
\endhead
R & RAM\tabularnewline
word & CPU\tabularnewline
excel & Processors\tabularnewline
\bottomrule
\end{longtable}

\hypertarget{rstudio}{%
\subsection{RStudio}\label{rstudio}}

But the problem is when somebody knows the interface (software that uses many types of software). Examples of this are programs like \texttt{RStudio}, \texttt{vs-code} and other such programs. This is because they are a single piece of software that uses ``heaps'' of other software.

\includegraphics{Simple_process_diagrams_files/figure-latex/unnamed-chunk-16-1.pdf}

\hypertarget{vs-code}{%
\subsection{`vs-code'}\label{vs-code}}

But then each bit of software such as \texttt{GIT} and \texttt{R} can be accessed by many different types of software\ldots so is \texttt{R} not just hardware then?

\includegraphics{Simple_process_diagrams_files/figure-latex/unnamed-chunk-17-1.pdf}

\hypertarget{r}{%
\subsection{R}\label{r}}

\includegraphics{Simple_process_diagrams_files/figure-latex/unnamed-chunk-18-1.pdf}

\hypertarget{computer-processes}{%
\section{Computer processes}\label{computer-processes}}

\hypertarget{rstudio-1}{%
\subsection{RStudio}\label{rstudio-1}}

This is an example of a piece of software. It needs your computer (hardware) to run so therefore it is software. Just the same as \texttt{R}\ldots.but not really.

\hypertarget{a-simple-rscript}{%
\subsubsection{\texorpdfstring{A simple \texttt{Rscript}}{A simple Rscript}}\label{a-simple-rscript}}

\includegraphics{Simple_process_diagrams_files/figure-latex/unnamed-chunk-19-1.pdf}

But this is not really what is happening. For example this is the \texttt{RMarkdown} figure for this process:

\includegraphics{Simple_process_diagrams_files/figure-latex/unnamed-chunk-20-1.pdf}

\hypertarget{v3}{%
\subsection{v3}\label{v3}}

\includegraphics{Simple_process_diagrams_files/figure-latex/unnamed-chunk-21-1.pdf}

\hypertarget{v4}{%
\subsection{v4}\label{v4}}

\includegraphics{Simple_process_diagrams_files/figure-latex/unnamed-chunk-22-1.pdf}

\hypertarget{sppPro}{%
\chapter{Interacting Species}\label{sppPro}}

To build diagrams that accounts for the issues and nature of data collection as well as incorperating the life history constrains of the species being modelled.

\begin{itemize}
\tightlist
\item
  \protect\hyperlink{exams}{Examples}
\end{itemize}

\hypertarget{three-species-models}{%
\section{Three species models}\label{three-species-models}}

\hypertarget{v1}{%
\subsection{v1}\label{v1}}

\includegraphics{../figs/unnamed-chunk-25-1.pdf}

\hypertarget{v2}{%
\subsection{v2}\label{v2}}

\includegraphics{../figs/unnamed-chunk-26-1.pdf}

\hypertarget{likscale}{%
\chapter{Likert Scales}\label{likscale}}

\hypertarget{the-problem}{%
\section{The problem}\label{the-problem}}

To build a sampling design that accounts for the issues and nature of data collection of foriegn bank movements using person-to-person == surveys.

\hypertarget{method}{%
\section{Method}\label{method}}

Here we want to address the key objectives of our reseach by collecting data in an unbias and representative way. To understand the true underlying trends in the data a pilot study with a large enough sample is explored below.

Here we have a difficult question to address due to the bias in the data capture method. This is really interesting and challenging from a statistical prospective. This is also a problem that the qualitative approach fails to address (waiting to see what others say about this..?)

Here is a step-by-step to approach to this research in a quantativite way:

\begin{enumerate}
\def\labelenumi{\arabic{enumi}.}
\tightlist
\item
  The questions are grouped into sets of questions that all address a general objective or question the research has. We can first see if these patterns are true for the responses. These are as follows:
\end{enumerate}

\includegraphics{../figs/unnamed-chunk-29-1.pdf}

There are several issues with this because the data collected is often non-normal and needs to be transformed. The first step however is to simply visualise the data in a way that we will also be fitting the statistical models.

\includegraphics{../figs/unnamed-chunk-30-1.pdf}

There is some simple dataset transformations that can be done to make it easier to work with the data in the plotting functions.

\includegraphics{../figs/unnamed-chunk-31-1.pdf}

\hypertarget{flow-diagram}{%
\section{Flow diagram}\label{flow-diagram}}

Here I think we should add the relationship to each question and the proposed hypothesis\ldots{}

\hypertarget{v1-1}{%
\subsection{v1}\label{v1-1}}

\includegraphics{../figs/unnamed-chunk-32-1.pdf}

\hypertarget{v2-1}{%
\subsection{v2}\label{v2-1}}

A similar automatic report can be generated using by using the DIY\_ANOVA app \href{https://pecostats.shinyapps.io/DIY_ANOVA/}{here}. It consists of a reduced version of the following report and sections.

\includegraphics{../figs/unnamed-chunk-33-1.pdf}

\includegraphics{../figs/unnamed-chunk-34-1.pdf} \includegraphics{../figs/unnamed-chunk-34-2.pdf} \includegraphics{../figs/unnamed-chunk-34-3.pdf} \includegraphics{../figs/unnamed-chunk-34-4.pdf} \includegraphics{../figs/unnamed-chunk-34-5.pdf} \includegraphics{../figs/unnamed-chunk-34-6.pdf}

\hypertarget{v3-1}{%
\subsection{v3}\label{v3-1}}

\includegraphics{../figs/unnamed-chunk-35-1.pdf}

\hypertarget{v4-1}{%
\subsection{v4}\label{v4-1}}

\includegraphics{../figs/unnamed-chunk-36-1.pdf}

\hypertarget{mixed}{%
\chapter{Mixed regression models}\label{mixed}}

\hypertarget{models}{%
\section{Models}\label{models}}

Mixed models essentially start with sample mean differences between groups and then add additional complexity in favour of poor fitting models.

You can write out models and variable relationships using statistical equations such as \(y_i = \beta_0 + \beta_1(x_i)\) where the variables can be collected either through direct questions or transformed data from the orginal questions. What does it look like in a diagram explaining how the variables are proposed to be similar? Example below:

\hypertarget{constant-only}{%
\subsection{Constant only}\label{constant-only}}

\[ outcome = \beta_0\]

\includegraphics{../figs/unnamed-chunk-37-1.pdf}

\hypertarget{single-predictor}{%
\subsection{Single predictor}\label{single-predictor}}

\[ outcome = \beta_0 + \beta_1(sex_i)\]

\includegraphics{../figs/unnamed-chunk-38-1.pdf}

\hypertarget{multi-level-v1}{%
\subsection{Multi-level v1}\label{multi-level-v1}}

For example you might think that generally individuals that \ldots{}

\[ outcome = \beta_0 + \beta_1(sex_i) + \beta(age_i) + error_i \]

\includegraphics{../figs/unnamed-chunk-39-1.pdf}

\hypertarget{multi-level-v2}{%
\subsection{Multi-level v2}\label{multi-level-v2}}

For example you might think that generally individuals that \ldots{}

\[ outcome = \beta_0 + \beta_1(sex_i) + \beta(age_i) + error_i \]

\includegraphics{../figs/unnamed-chunk-40-1.pdf}

\hypertarget{nzbs}{%
\chapter{NZ Beech Forests}\label{nzbs}}

The first beech forest paper assumes and tests the following process:

\[log(N) = Control + Valley + Rats + Control:Rats\]
\[r_{j,t} = \beta_{0,s,v,c} + \beta_{1,s,v,c} (S_{j,t})+ \beta_{2,s,v,c} (N_{j,t-1}) + \beta_{3,s,v,c} (R_{j,t-1})\]

\hypertarget{draft-ideas}{%
\section{Draft ideas}\label{draft-ideas}}

\hypertarget{v1-2}{%
\subsection{v1}\label{v1-2}}

\includegraphics{../figs/unnamed-chunk-41-1.pdf}

\hypertarget{v2-2}{%
\subsection{v2}\label{v2-2}}

\includegraphics{../figs/unnamed-chunk-42-1.pdf}

\hypertarget{v3-2}{%
\subsection{v3}\label{v3-2}}

\includegraphics{../figs/unnamed-chunk-43-1.pdf}

\hypertarget{mpd}{%
\chapter{NZ Mixed Forests}\label{mpd}}

\[log(N) = Control + Valley + Rats + Control:Rats\]
\[r_{j,t} = \beta_{0,s,v,c} + \beta_{1,s,v,c} (S_{j,t})+ \beta_{2,s,v,c} (N_{j,t-1}) + \beta_{3,s,v,c} (R_{j,t-1})\]

The first beech forest paper assumes and tests the following model.

\hypertarget{draft-ideas-1}{%
\section{Draft ideas}\label{draft-ideas-1}}

\hypertarget{v1-3}{%
\subsection{v1}\label{v1-3}}

\includegraphics{../figs/unnamed-chunk-44-1.pdf}

\hypertarget{v2-3}{%
\subsection{v2}\label{v2-3}}

\includegraphics{../figs/unnamed-chunk-45-1.pdf}

\hypertarget{v3-3}{%
\subsection{v3}\label{v3-3}}

\includegraphics{../figs/unnamed-chunk-46-1.pdf}

\hypertarget{draft-processes}{%
\section{Draft processes}\label{draft-processes}}

\hypertarget{mice}{%
\subsection{Mice}\label{mice}}

\hypertarget{mice-and-rats}{%
\subsection{Mice and Rats}\label{mice-and-rats}}

\hypertarget{mice-rats-and-possums}{%
\subsection{Mice, rats and possums}\label{mice-rats-and-possums}}

\hypertarget{mice-rats-possums-and-stoats}{%
\subsection{Mice, rats, possums and stoats}\label{mice-rats-possums-and-stoats}}

\bibliography{book.bib,packages.bib}


\end{document}
